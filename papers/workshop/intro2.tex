\section{Introduction}
\label{sec:introduction}

To this day, {\em randomized experiments} (A/B testing)
remain the gold standard for causal inference.
However, controlled experiments are not feasible or ethical,
for economical or practical reasons \cite{rosenbaum2002observational}.
Fortunately, {\em Observational studies} can be used to draw
causal inference
\cite{rosenbaum2002observational,Rubin2005,PearlBook2000\ignore{,Spirtes:book01}}.
Many fields (computational, physical, and social scientists)
want to perform causal inference on big data
from observational studies:
social networks, biological networks, sensor networks and more.
Unfortunately, current software for processing observational data
for causal inference does not scale.
R, Stata, SAS, and SPSS have packages such as
{\em MatchIt} and {\em CEM}\cite{ho2005,iacus2009cem},
but they are designed for single-table data and are cumbersome
with large datasets. For example, we found performing CEM on a
dataset with 5M entries takes up to an hour using Stata, R or SAS.


Additionally, causal analysis is part of a larger pipeline that includes
data acquisition, cleaning, and integration.
For large datasets, these tasks are better handled by
a relational database engine.


For this workshop, we propose a causal inference library for PostgreSQL.
This takes the first step towards truly scalable causal inference
by modeling it as a data management problem.
We demonstrate that causal inference can be approached from this perspective,
and that doing so is key to scalability and robustness.
To demonstrate the library's capabilities, we will run causal analysis
on two datasets.

\begin{example} \em Lalonde Dataset
\label{ex:lalonde}


Our first dataset is from the National Supported Work Demonstration,
a U.S. job training program.
\corey{cite lalonde 1986}
The purpose of the program (treatment) was to increase participants' income,
thus 1978 earnings (re78) is the outcome variable.
Lalonde (1986) generated both an experimental and observational dataset
which are still used to this day to benchmark observational causal inference methods.
This dataset contains less than 1000 entries.
Throughout the paper, we will use this dataset as an example to
explain observational causal inference techniques at a high level.
\end{example}

\begin{example} \em Flight Weather Merged Dataset
\label{ex:flights}


Our second dataset demonstrates the capabilities of bringing causal inference
into the DBMS.
We acquired flight departure details for all commercial flights
within the US from 2000 to 2015 (105M entries)
and combined this with historical weather data (35M entries).
These are relatively large data sets for causal inference and cannot be handled
by existing tools.
We will use this dataset in our experimental results section to demonstrate
the scalability of our library.
\end{example}

\corey{I think we can skip the tables of attributes} 