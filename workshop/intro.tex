\section{Introduction}
%\dan{7 pages}
\label{sec:introduction}

%\dan{this part is mostly done, only minor edits needed}

% Big data is used today in a wide range of domains, such as natural
% sciences (physics CITE, astronomy CITE, genomics CITE, biology CITE),
% social science and in particular measurements of human activity
% (e.g. recommendation CITE, personalization CITE),
% education~\cite{clauset-2015}, marketing and economics CITE, and MORE
% HERE.  The most successful type of application of Big data is {\em
%   predictive}: the data available, the \emph{seen} data, is used to
% infer a model that can predict features and trends of data that is not
% available, the \emph{unseen} data.  This potential for predictive
% analysis has generated the huge interest we see today in Big data, and
% has, in some sense, democratized data, by stimulating a huge number of
% ``data enthusiasts'' to collect, explore, analyze, integrate, and
% visualize data.  The term Big data is often used today to refer not
% just to the traditional features like volume, variety, velocity, but
% also to its wide availability to a broad range of users.


% lg1 - original commentout below at end; take care as this can lead to confusion

Much of the success of Big data today comes from \emph{predictive or descriptive
  analytics}:  statistical models or data mining algorithms applied to data
to predict new or future observations, e.g., we observe
how users click on ads, then build a model and predict how future
users will click.  Predictive analysis/modeling is central to many scientific fields, such as
bioinformatics and natural language processing, in other  fields - such as social economics, psychology, education and environmental
science - researchers are focused on testing and evaluating {\em causal hypotheses}. While the distinction between causal
and predictive analysis has been recognized, the conflation between the two is common.


Causal inference has been studied extensively in statistics and
computer science \cite{Fisher1935design,Rubin2005,holland1986statistics,PearlBook2000\ignore{,Spirtes:book01}}.
Many tools perform causal inference
 using statistical software such as SAS, SPSS, or R project. However, these toolkits do not scale to large datasets.  Furthermore, in many of the most interesting Big Data settings, the data
is highly relational (e.g, social networks, biological networks, sensor networks and more) and likely to pour into SQL systems. There is a rich ecosystem of tools and organizational requirements that encourage this. Transferring
data from DBMS to statistical softwares or connecting these
softwares to DBMS can be error prone, difficult, time consuming and inefficient. For these
cases, it would be helpful to push statistical methods for causal inference into the DBMS.



Both predictive and causal analysis are needed to generate
and test theories,  policy and decision making and to evaluate hypotheses, yet each plays a different role in doing so. In fact, performing predictive analysis to address questions that are causal in nature could lead to a flood of false discovery claims. In many cases, researchers who want to discover causality from data analysis settle for predictive analysis either because they think it is causal or lack of available alternatives.

This work introduces \GSQL,\footnote{ The prefix Zali refers to
  al-Ghzali (1058-1111), a medieval Persian philosopher. It is known
  that David Hume (1711-1776), a Scottish philosopher, who gave the
  first explicit definition of causation in terms of counterfactuals,
  was heavily influenced by al-Ghzali's conception of causality
  \cite{shalizi2013advanced}.}  a SQL-based framework for drawing
causal inference that circumvents the scalability issue with the
existing tools.  ZaliQL supports state-of-the-art methods for causal
inference and runs at scale within a database engine.  We show how to
express the existing advanced causal inference methods in SQL, and
develop a series of optimization techniques allowing our system to
scale to billions of records. We evaluate our system on a real
dataset.  Before describing the contributions of this paper, we
illustrate causal inference on the following real example.

\begin{table*}[!htb] \scriptsize
    \begin{subtable}{.5\linewidth}
      \centering

       \begin{tabular}[t]{|l|l|}
  \hline
  % after \\: \hline or \cline{col1-col2} \cline{col3-col4} ...
  \bf{Attribute}   & \bf{Description}  \\  \hline
  FlightDate & Flight date   \\ \hline
  UniqueCarrier	& Unique carrier code   \\ \hline
  OriginAirportID & 	Origin airport ID  \\ \hline
  CRSDepTime & Scheduled departure time  \\ \hline
  DepTime & Actual departure time \\ \hline
    & difference in minutes between   \\
  DepDelayMinutes & scheduled and actual departure \\
   & time. Early departures set to 0  \\ \hline
  LateAircraftDelay & Late aircraft delay, in minutes  \\ \hline
  SecurityDelay & Ssecurity delay, in minutes \\ \hline
  CarrierDelay & Carrier delay, in minutes  \\ \hline
  Cancelled & Binary indicator  \\
  \hline
\end{tabular}
     \caption{Flight dataset}
    \end{subtable}%
    \begin{subtable}{.5\linewidth}
      \centering


       \begin{tabular}[t]{|l|l|}
  \hline
  % after \\: \hline or \cline{col1-col2} \cline{col3-col4} ...
  \bf{Attribute}   & \bf{Description}  \\  \hline
  Code  & Airport ID    \\ \hline
  Date	& Date of a repost   \\ \hline
  Time        & Time of a report  \\ \hline
  Visim        & Visibility in km  \\ \hline
  Tempm & Temperature in C$^{\circ}$ \\ \hline
  Wspdm         & Wind speed kph \\ \hline
  \ignore{Precipm         & Humidity \% \\ \hline}
  Pressurem         & Pressure in mBar  \\ \hline
  Precipm         & Precipitation in mm  \\ \hline
  \ignore{Rain & Binary indictor    \\ \hline
  Snow & Binary indictor \\ \hline}
  Tornado & Binary indictor \\ \hline
  Thunder & Binary indictor \\ \hline
  Hum & Humidity \% \\ \hline
  Dewpoint & De point in  C$^{\circ}$ \\ \hline
\end{tabular}
        \caption{Weather dataset}
    \end{subtable}
 \vspace{-0.1cm}   \caption{\bf{List of attributes from the flight(a) and weather(b)  datasets that are relevant to our analysis.}}
\label{tab:attlist}
\end{table*}
\vspace{-0.1cm}

\babak{Add another dataset which is more close in nature to causal inference, i.e., a dataset with more tangible treatment, e.g.,  Lalonde data or a medical treatment}

\babak{We should look into the CEM matching paper for the flow of introduction
\url{https://cran.r-project.org/web/packages/cem/vignettes/cem.pdf}}

\vspace{-.2cm}
\begin{example} \em \delay. \ \em   Flight delays pose a serious
and widespread problem in the United States and
 significantly strain on the national air travel system, costing society many billions of dollars each year \cite{ball2010total}.
 According to FAA statistics,\footnote{National Aviation Statistic \url{http://www.faa.gov/}}
   weather causes approximately 70\% of the delays in the US National Airspace System (NAS). The upsetting impact
   of weather conditions on aviation is well known, however quantifying
  the causal impact of different weather types on flight delays
at different airports is essential for evaluating
approaches to reduce these delays. Even though predictive analysis, in this context, might help make certain policies,
this problem is causal. We conduct this causal analysis as a running example through this paper. To this end,
we acquired flight departure details for all commercial flights within the US from 2000 to 2015
(105M entries) and integrated it with the relevant historical weather data (35M entries) (see Section \ref{sec:setup}).
These are relatively large data sets for causal inference that can not be handseled by the existing tools. Table \ref{tab:attlist} presents
the list of attributes from each data set that is relevant to our analysis.







\ignore{
We argue that this problem is causal in nature and performing
a predictive analysis can be very misleading. To conduct the analysis, we collect the flight and weather data
since 2005.


The flight data is acquired from the US Department of
Transportation [117] and consists of 168 million rows. The
weather data is gathered using the weather underground
API
2
and consists of 10 million rows


Suppose we are interested in the impact of
low pressure on flight departure delay at X airport.


 To highlight the distinction between predictive and causal nalsysis


Suppose we are interested to see if weather low pressure
has any impact on departure delay at X airport. By conducting a
}


\ignore{
This analysis is causal in nature. For example,
when we concern about the effect of low pressure on flight
departure delay, we are not interested in the difference
 between average flights departure delay when low
 barometric pressure is reported at the flight
 time and the average flight departure delay when high
  pressure is reported. Rather, researchers are interested
  to find out weather low/high barometric pressure has any causal
  impact on flight departure delay and if so, to compute the strength of that effect.

To say it in a somewhat different way, the observed distribution of
the flight departure delay, for instance,  at John F. Kennedy International Airport (JFK) in 2015 is an outcome of a complicated stochastic process. When we make
a probabilistic prediction of the departure delay, $Y$,  by
conditioning on low/high barometric pressure,\ignore{\footnote{ Barometric pressure above 1022.69 and below  1009.14 millibar is usually considered as high and low pressure respectively \cite{barometricpressureheadache:article}.}} $X$ (low barometric pressure (=1) vs. hight barometric pressure (=0))- whether we predict $\E[Y | X = x]$ or
$Pr(Y | X = x)$, with $x \in \{0,1\}$, or something more
complicated- we are just filtering the output of
the mechanisms that controlled the flight departure
 delay at JFK in 2015, picking out the cases where they
  happen to have set $X$ to the value $x$, and looking
  at what goes along with that. In fact, by applying this
   naive predictive analysis, we obtain that
   $E(Y | X = 1)-E(Y | X = 0)\backsimeq 4$, which suggests that
    barometric pressure affected the flight departure delay at
    JFK in 2015 and might be a good predictor for that.}
\end{example}





When we make predictive analysis, whether we predict $\E[Y | X = x]$ or $\textrm{Pr}(Y | X = x)$ or
something more complicated, we essentially want to know the conditional distribution of
$Y$ given $X$. On the other hand, when we make a causal
analysis, we want to understand the distribution of $Y$, if the
usual mechanisms controlling $X$ were intervened and set to $x$.
In other words, in causal analysis we are interested in {\em  interventional} conditional
distribution, e.g.,  the distribution  obtained by (hypothetically)
enforcing $X = x$ uniformly over the population.  In causal analysis, the difficulty arises   from the fact that here the objective is to estimate (unobserved)
   {\em counterfactuals} from the (observed) {\em factual} premises.



   \begin{example} \em \delay \ (Cont.).  \label{ex:press} \em Suppose
     we want to explore the effect of low-pressure on flight departure
     delays. High pressure is generally associated with clear weather,
     while low-pressure is associated with unsettled weather, e.g.,
     cloudy, rainy, or snowy
     weather\ignore{\cite{weba2,barometricpressureheadache:article}}. Therefore,
     conducting any sort of predictive analysis identifies
     low-pressure as a predictor for flight delays. However,
     low-pressure does not have any causal impact on departure delay
     (low-pressure only requires longer takeoff distance)
     \cite{FAA08}.  That is, low-pressure is most highly a correlated
     attribute with flight delays, however ZaliQL found that other
     attributes such as thunder, low-visibility, high-wind-speed and
     snow have the largest causal effect on flight delays (see
     Sec. \ref{sec:endtoend}); this is confirmed by the results
     reported by the FAA and \cite{weather}.

\end{example}






% They defined a very simple model, where we
% want to conclude \dan{Lise: what's your comment here?}  if one
% variable, called ``treatment'' causes one particular output variable,
% called ``effect'', and have developed a rich set of techniques for
% that purpose.  The end goal of their analysis is to establish the
% average causal-treatment effect.  More recently, in the AI literature
% Pearl~\cite{pearl2010introduction,PearlBook2000} has extended this
% simple model by introducing {\em causal networks}, and developing a
% logical framework for reasoning about causality.  Their aim is to
% enable complex inference in a network of causal associations.  While
% the ultimate goal is the same, to check whether a particular input
% causes a particular outcome, the methods deployed are different from
% those in statistics.

\vspace{-0.1cm}

This paper describes novel techniques implementing and optimizing
state-of-the-art causal inference methods (reviewed in Section
\ref{subsec:causalitystatistics}) in relational databases.  We make
three contributions.  First, in Section \ref{sec:BasicTechniques} we
describe the basic relational implementation of the main causal
inference methods: matching and subclassification.  Second, in Section
\ref{sec:OptimizationTechniques} we describe a suite of optimization
techniques for subclassfication, both in the online and offline
setting.  Finally, in Section \ref{sec:exp} we conduct an extensive
empirical evaluation of \GSQL, our system that implements these
techniques, on real data from the U.S. DOT and Weather Underground
\cite{flightdata,Weatherdata}.


\ignore{
This paper makes the following specific contributions: we describe a
suite of techniques for expressing the existing advanced methods for
causal inference from observational data in SQL that run at scale
within a database engine (Section \ref{sec:BasicTechniques}). Note
that we do not claim any contribution the existing methods; we
introduce several optimization techniques that significantly speedup
causal inference, both in the online and offline setting (Section
\ref{sec:OptimizationTechniques}); we validate our system
experimentally, using real data from the U.S. DOT and Weather
Underground \cite{flightdata,Weatherdata}
}

\vspace{-2mm}
