\vspace{-.42cm}
\section{Related work and conclusion}
%\dan{7 pages}
\label{sec:rel}

\ignore{
The study of causality in statistics can be traced back to Neyman~\cite{neyman1923},
Fisher~\cite{Fisher1935design}, Rubin~\cite{Rubin1974,Rubin2005} and
Holland~\cite{Holland1986}, whose theories  led to the {\em potential outcome framework} for causal
inference. More recently, Pearl~\cite{PearlBook2000}, Spirtes et al.~\cite{Spirtes:book01} and
others have developed causal graphical models, which provide a graphical way to describe
causal relationships and support reasoning over multiple causal relationships. However \ignore{
In both cases, causal inference  critically depends on the precise methodology and on key assumptions about the data,
without which causal relationships cannot be claimed.}  See \cite{richardson2013single} for a connection between these two approaches.}
The simple nature of the R, and its adherence to a few statistical assumptions, makes it more appealing for the researchers. Therefore,
 it has become the  prominent approach in social sciences, biostatistics, political science, economics
 and other disciplines. Many toolkits have been developed for performing casual inference {\em a` la} this framework that
 depends on  statistical software such as SAS, SPSS, or R project. However, these toolkits do not scale to large datasets. This work introduce  \GSQL,  a SQL-based framework for drawing causal
  inference that circumvents the scalability issue with the existing
  tools. ZaliQL supports state-of-the-art methods for causal
  inference and runs at scale within a database engine.


The notion of causality has been studied extensively in databases \cite{MeliouGMS2011,RoyS14,DBLP:conf/icdt/SalimiB15,SalimiTaPP16}.  We note that this line of work is different than the present paper in the sense that, it aims to identify causes for an observed output of a data transformation. \ignore{For example, in query-answer causality/explanation, the objective is to identify parts of a database that causally explain a result of a query. } While these works share some aspects of the notion of causality as studied in this paper, the problems that they address are fundamentally different. 