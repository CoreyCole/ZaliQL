<<<<<<< HEAD
\section{The \GSQL \ System}
=======
\section{The Viska System}

We are developing a system/toolkit called Viska, which is aimed at automating
and guiding the three core tasks of performance analysis: generating,
exploring and analyzing performance data.

The primary contribution of our toolkit is a set of tools to support
data analysis.
This includes a web-based interactive data visualization app, and a
Python library that makes it possible to leverage existing data
analytics and machine learning platform, including Jupyter Notebook
and scikit-learn.
Additionally, we are developing a
library that enables hypothesis testing through state-of-the-art causal inference
techniques. % statistical validity, rigor, ...

\subsection{Generating performance data}
Experimental design. Input variables, observed variables, outcome variables.
Viska allows you to define these and use as parameters in exploration and
analyzing.

\subsection{Exploring performance data}
Python library and web app. Exploring data to find correlations, interesting
behavior. Preparing data for further analysis: removing unnecessary features,
merging features, bucketing, etc.

We have developed an interactive visualization
tool that supports exploratory analysis of complex performance
data. Drawing inspiration from the principles of exploratory
data analysis and various visualization techniques,
we provide an engaging way for users to explore performance
results. The system aims to give a clear overview
of available data, while focusing the attention of the users to
select subsets of features and data to be combined into a single
customized dashboard targeted to the hypothesis being
investigated.

\subsection{Analyzing performance results}
% What is the causal relationship between two features, the \emph{treatment} 
% (cause) and the \emph{outcome} (effect). Causal inference basics. Need to
% adjust for confounding effects of other variables. Consider relevant variables
% as covariates in causal inference. Hard to control for everything, so we
% use graph generation algorithms to help with covariate selection. 
% Hypothesis testing, statistical validation.
Interactive visualization makes it easier to discover correlations
between workload parameters, system metrics, and performance results
-- but do these imply causation's? Viska's
interactive visualization interface integrates a new library we
developed for statistically sound hypothesis testing.

Viska's hypothesis testing is based on \emph{causality inference},
using the Neyman-Rubin model. At a high level, causality inference
measures the influence of one variable (the \emph{treatment})
on an outcome, isolating it from other confounding factors.

%How causality inference works...
In this model, we select a single binary attribute $T$ as the
\emph{treatment}.  For each unit (i.e., experiment run), the unit is either
treated ($T = 1$) or subjected to control ($T = 0$).  Another attribute $Y$ is
selected as the \emph{outcome}; $Y(0), Y(1)$ represent \emph{potential
outcomes} when the unit is exposed to the treatment ($Y(1)$) or the control
($Y(0)$). As an example, we could select query duration as our outcome, and
whether or not the query is an aggregate as our treatment.
A vector $X$ represents potentially confounding variables, called \emph{covariates}.
These can be attributes such as the number of input and output rows.

The goal is to compute the \emph{average treatment effect (ATE)}:
$$
\text{ATE} = \mathbb{E}[Y(1) - Y(0)] = \mathbb{E}[Y(1)] - \mathbb{E}[Y(0)] 
$$
% For example, our experiments show a correlation between queries containing
% aggregates and Spark outperforming Myria. However, the ATE computation reveals
% that this is spurious: the performance difference is actually associated with a
% covariate. Queries with aggregates have fewer output rows, and it is on these
% that Spark outperforms Myria.

The challenge in computing ATE is that the model assumes that both
$Y(1)$ and $Y(0)$ are available for each unit, which is not the case
in reality.
Viska's analysis library uses several techniques, including 
\emph{coarsened exact matching (CEM)}, which
prunes results so that the remaining data achieves \emph{balance}
between the treated and control groups, and thus controls for the confounding
influence of the covariates.
>>>>>>> 61cbea7c490bf41aa51f4cc1691b2ed66a6001eb
