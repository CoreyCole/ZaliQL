\section{Demonstration Proposal}
\label{sec:dd}

The demonstration starts with weather and flight data.

\ignore{
\begin{enumerate}
\item {\it Selecting a causal hypothesis:}
  the attendee selects a causal hypothesis to test by specifying a treatment (potential cause)
  and an outcome (potential effect) (cf, Example \ref{sfig:testaa}).
\item {\it Matching method:} The user select a matching method to
\item Enable attendees to drill into the details of how we generate these
results (i.e.\, the different steps of the pipeline).
\item Show what happens for different configurations of the Viska system:
e.g.\, different parameters to graph-generation algorithms, different bucketing
strategies, etc.
\end{enumerate}
}

\begin{enumerate}
  \item {\it Form causal hypotheses:} the attendee forms a set of causal hypotheses, e.g.,
   e.g. weather low-visibility causes fight delays at John F. Kennedy International Airport(JFK), by
    \begin{enumerate}
      \item specifying a set of binary treatments, e.g., low-visibility, Snow, ... (potential causes)
      \item specifying an outcome of interest (potential effect) e.g., departure delay, cancelation, ..
      \item specifying a subset of data that relevant to the analysis with e.g., flights at JFK airport
    \end{enumerate}

  \item {\it Setup inference parameters:} The user select and tune the matching algorithms as follows:
     \begin{enumerate}
      \item For each treatment the user the user specifics a set of variables that judged to confound it with the associated outcome.
          \item The user specifies a matching methods supported by \GSQL\ and adjust its tuning parameters
\end{enumerate}

  \item {\it View matching assessment:} (cf, Example \ref{sfig:testbb})
    \begin{enumerate}
      \item The user asses how the matching methods was successful by checking the numerical summaries and plots provided by \GSQL \ignore{  balance of covariates after matching, including numerical summaries such as the mean Diff. (difference in means), and summaries based on quantile quantile plots and multivariate histograms that compare the empirical distributions of each covariate} as shown in Figure \ref{fig:inteface}(c).
      \item
      if the matched data associated to  particular treatment is still imbalance, the user can re-run the procedure, for the treatment,
        using a different matching method or by tuning the relevant parameters (i.e. Thunder in \ref{sfig:testbb})
    \end{enumerate}
  \item {\it Hypothesis testing:}
    \begin{enumerate}
      \item The user can select and perform a preferred statistical methods for hypothesis testing, e.g., Chi-square, test on the matched data.
    \item {\it Analysis of performance: We compare the performance of \GSQL\ with other statistical software such as R.}
   
    \end{enumerate}
\end{enumerate}



\ignore{
\paragraph{Performance Data.}

Viska generates and analyzes performance data obtained by benchmarking
systems. This performance data consists of a set of records
each representing a single execution of an experiment. This record has
a set of ``input'' variables that we vary between experiments, such as
type of queries run, data set used, or configuration parameters. Viska
records these along with a set of ``observed'' variables: properties
of the execution (e.g., number of rows processed), system-level
metrics (e.g., CPU and disk utilization), and outcome variables (e.g.,
query execution time).

The Viska toolkit includes tools to generate and analyze this
performance data. We do not describe the tools for workload
generation, system deployment, and metric gathering here.
}
